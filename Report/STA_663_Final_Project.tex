
% Default to the notebook output style

    


% Inherit from the specified cell style.




    
\documentclass{article}

    
    
    \usepackage{graphicx} % Used to insert images
    \usepackage{adjustbox} % Used to constrain images to a maximum size 
    \usepackage{color} % Allow colors to be defined
    \usepackage{enumerate} % Needed for markdown enumerations to work
    \usepackage{geometry} % Used to adjust the document margins
    \usepackage{amsmath} % Equations
    \usepackage{amssymb} % Equations
    \usepackage{eurosym} % defines \euro
    \usepackage[mathletters]{ucs} % Extended unicode (utf-8) support
    \usepackage[utf8x]{inputenc} % Allow utf-8 characters in the tex document
    \usepackage{fancyvrb} % verbatim replacement that allows latex
    \usepackage{grffile} % extends the file name processing of package graphics 
                         % to support a larger range 
    % The hyperref package gives us a pdf with properly built
    % internal navigation ('pdf bookmarks' for the table of contents,
    % internal cross-reference links, web links for URLs, etc.)
    \usepackage{hyperref}
    \usepackage{longtable} % longtable support required by pandoc >1.10
    \usepackage{booktabs}  % table support for pandoc > 1.12.2
    

    
    
    \definecolor{orange}{cmyk}{0,0.4,0.8,0.2}
    \definecolor{darkorange}{rgb}{.71,0.21,0.01}
    \definecolor{darkgreen}{rgb}{.12,.54,.11}
    \definecolor{myteal}{rgb}{.26, .44, .56}
    \definecolor{gray}{gray}{0.45}
    \definecolor{lightgray}{gray}{.95}
    \definecolor{mediumgray}{gray}{.8}
    \definecolor{inputbackground}{rgb}{.95, .95, .85}
    \definecolor{outputbackground}{rgb}{.95, .95, .95}
    \definecolor{traceback}{rgb}{1, .95, .95}
    % ansi colors
    \definecolor{red}{rgb}{.6,0,0}
    \definecolor{green}{rgb}{0,.65,0}
    \definecolor{brown}{rgb}{0.6,0.6,0}
    \definecolor{blue}{rgb}{0,.145,.698}
    \definecolor{purple}{rgb}{.698,.145,.698}
    \definecolor{cyan}{rgb}{0,.698,.698}
    \definecolor{lightgray}{gray}{0.5}
    
    % bright ansi colors
    \definecolor{darkgray}{gray}{0.25}
    \definecolor{lightred}{rgb}{1.0,0.39,0.28}
    \definecolor{lightgreen}{rgb}{0.48,0.99,0.0}
    \definecolor{lightblue}{rgb}{0.53,0.81,0.92}
    \definecolor{lightpurple}{rgb}{0.87,0.63,0.87}
    \definecolor{lightcyan}{rgb}{0.5,1.0,0.83}
    
    % commands and environments needed by pandoc snippets
    % extracted from the output of `pandoc -s`
    \DefineVerbatimEnvironment{Highlighting}{Verbatim}{commandchars=\\\{\}}
    % Add ',fontsize=\small' for more characters per line
    \newenvironment{Shaded}{}{}
    \newcommand{\KeywordTok}[1]{\textcolor[rgb]{0.00,0.44,0.13}{\textbf{{#1}}}}
    \newcommand{\DataTypeTok}[1]{\textcolor[rgb]{0.56,0.13,0.00}{{#1}}}
    \newcommand{\DecValTok}[1]{\textcolor[rgb]{0.25,0.63,0.44}{{#1}}}
    \newcommand{\BaseNTok}[1]{\textcolor[rgb]{0.25,0.63,0.44}{{#1}}}
    \newcommand{\FloatTok}[1]{\textcolor[rgb]{0.25,0.63,0.44}{{#1}}}
    \newcommand{\CharTok}[1]{\textcolor[rgb]{0.25,0.44,0.63}{{#1}}}
    \newcommand{\StringTok}[1]{\textcolor[rgb]{0.25,0.44,0.63}{{#1}}}
    \newcommand{\CommentTok}[1]{\textcolor[rgb]{0.38,0.63,0.69}{\textit{{#1}}}}
    \newcommand{\OtherTok}[1]{\textcolor[rgb]{0.00,0.44,0.13}{{#1}}}
    \newcommand{\AlertTok}[1]{\textcolor[rgb]{1.00,0.00,0.00}{\textbf{{#1}}}}
    \newcommand{\FunctionTok}[1]{\textcolor[rgb]{0.02,0.16,0.49}{{#1}}}
    \newcommand{\RegionMarkerTok}[1]{{#1}}
    \newcommand{\ErrorTok}[1]{\textcolor[rgb]{1.00,0.00,0.00}{\textbf{{#1}}}}
    \newcommand{\NormalTok}[1]{{#1}}
    
    % Define a nice break command that doesn't care if a line doesn't already
    % exist.
    \def\br{\hspace*{\fill} \\* }
    % Math Jax compatability definitions
    \def\gt{>}
    \def\lt{<}
    % Document parameters
    \title{STA\_663\_Final\_Project}
    
    
    

    % Pygments definitions
    
\makeatletter
\def\PY@reset{\let\PY@it=\relax \let\PY@bf=\relax%
    \let\PY@ul=\relax \let\PY@tc=\relax%
    \let\PY@bc=\relax \let\PY@ff=\relax}
\def\PY@tok#1{\csname PY@tok@#1\endcsname}
\def\PY@toks#1+{\ifx\relax#1\empty\else%
    \PY@tok{#1}\expandafter\PY@toks\fi}
\def\PY@do#1{\PY@bc{\PY@tc{\PY@ul{%
    \PY@it{\PY@bf{\PY@ff{#1}}}}}}}
\def\PY#1#2{\PY@reset\PY@toks#1+\relax+\PY@do{#2}}

\expandafter\def\csname PY@tok@gd\endcsname{\def\PY@tc##1{\textcolor[rgb]{0.63,0.00,0.00}{##1}}}
\expandafter\def\csname PY@tok@gu\endcsname{\let\PY@bf=\textbf\def\PY@tc##1{\textcolor[rgb]{0.50,0.00,0.50}{##1}}}
\expandafter\def\csname PY@tok@gt\endcsname{\def\PY@tc##1{\textcolor[rgb]{0.00,0.27,0.87}{##1}}}
\expandafter\def\csname PY@tok@gs\endcsname{\let\PY@bf=\textbf}
\expandafter\def\csname PY@tok@gr\endcsname{\def\PY@tc##1{\textcolor[rgb]{1.00,0.00,0.00}{##1}}}
\expandafter\def\csname PY@tok@cm\endcsname{\let\PY@it=\textit\def\PY@tc##1{\textcolor[rgb]{0.25,0.50,0.50}{##1}}}
\expandafter\def\csname PY@tok@vg\endcsname{\def\PY@tc##1{\textcolor[rgb]{0.10,0.09,0.49}{##1}}}
\expandafter\def\csname PY@tok@m\endcsname{\def\PY@tc##1{\textcolor[rgb]{0.40,0.40,0.40}{##1}}}
\expandafter\def\csname PY@tok@mh\endcsname{\def\PY@tc##1{\textcolor[rgb]{0.40,0.40,0.40}{##1}}}
\expandafter\def\csname PY@tok@go\endcsname{\def\PY@tc##1{\textcolor[rgb]{0.53,0.53,0.53}{##1}}}
\expandafter\def\csname PY@tok@ge\endcsname{\let\PY@it=\textit}
\expandafter\def\csname PY@tok@vc\endcsname{\def\PY@tc##1{\textcolor[rgb]{0.10,0.09,0.49}{##1}}}
\expandafter\def\csname PY@tok@il\endcsname{\def\PY@tc##1{\textcolor[rgb]{0.40,0.40,0.40}{##1}}}
\expandafter\def\csname PY@tok@cs\endcsname{\let\PY@it=\textit\def\PY@tc##1{\textcolor[rgb]{0.25,0.50,0.50}{##1}}}
\expandafter\def\csname PY@tok@cp\endcsname{\def\PY@tc##1{\textcolor[rgb]{0.74,0.48,0.00}{##1}}}
\expandafter\def\csname PY@tok@gi\endcsname{\def\PY@tc##1{\textcolor[rgb]{0.00,0.63,0.00}{##1}}}
\expandafter\def\csname PY@tok@gh\endcsname{\let\PY@bf=\textbf\def\PY@tc##1{\textcolor[rgb]{0.00,0.00,0.50}{##1}}}
\expandafter\def\csname PY@tok@ni\endcsname{\let\PY@bf=\textbf\def\PY@tc##1{\textcolor[rgb]{0.60,0.60,0.60}{##1}}}
\expandafter\def\csname PY@tok@nl\endcsname{\def\PY@tc##1{\textcolor[rgb]{0.63,0.63,0.00}{##1}}}
\expandafter\def\csname PY@tok@nn\endcsname{\let\PY@bf=\textbf\def\PY@tc##1{\textcolor[rgb]{0.00,0.00,1.00}{##1}}}
\expandafter\def\csname PY@tok@no\endcsname{\def\PY@tc##1{\textcolor[rgb]{0.53,0.00,0.00}{##1}}}
\expandafter\def\csname PY@tok@na\endcsname{\def\PY@tc##1{\textcolor[rgb]{0.49,0.56,0.16}{##1}}}
\expandafter\def\csname PY@tok@nb\endcsname{\def\PY@tc##1{\textcolor[rgb]{0.00,0.50,0.00}{##1}}}
\expandafter\def\csname PY@tok@nc\endcsname{\let\PY@bf=\textbf\def\PY@tc##1{\textcolor[rgb]{0.00,0.00,1.00}{##1}}}
\expandafter\def\csname PY@tok@nd\endcsname{\def\PY@tc##1{\textcolor[rgb]{0.67,0.13,1.00}{##1}}}
\expandafter\def\csname PY@tok@ne\endcsname{\let\PY@bf=\textbf\def\PY@tc##1{\textcolor[rgb]{0.82,0.25,0.23}{##1}}}
\expandafter\def\csname PY@tok@nf\endcsname{\def\PY@tc##1{\textcolor[rgb]{0.00,0.00,1.00}{##1}}}
\expandafter\def\csname PY@tok@si\endcsname{\let\PY@bf=\textbf\def\PY@tc##1{\textcolor[rgb]{0.73,0.40,0.53}{##1}}}
\expandafter\def\csname PY@tok@s2\endcsname{\def\PY@tc##1{\textcolor[rgb]{0.73,0.13,0.13}{##1}}}
\expandafter\def\csname PY@tok@vi\endcsname{\def\PY@tc##1{\textcolor[rgb]{0.10,0.09,0.49}{##1}}}
\expandafter\def\csname PY@tok@nt\endcsname{\let\PY@bf=\textbf\def\PY@tc##1{\textcolor[rgb]{0.00,0.50,0.00}{##1}}}
\expandafter\def\csname PY@tok@nv\endcsname{\def\PY@tc##1{\textcolor[rgb]{0.10,0.09,0.49}{##1}}}
\expandafter\def\csname PY@tok@s1\endcsname{\def\PY@tc##1{\textcolor[rgb]{0.73,0.13,0.13}{##1}}}
\expandafter\def\csname PY@tok@sh\endcsname{\def\PY@tc##1{\textcolor[rgb]{0.73,0.13,0.13}{##1}}}
\expandafter\def\csname PY@tok@sc\endcsname{\def\PY@tc##1{\textcolor[rgb]{0.73,0.13,0.13}{##1}}}
\expandafter\def\csname PY@tok@sx\endcsname{\def\PY@tc##1{\textcolor[rgb]{0.00,0.50,0.00}{##1}}}
\expandafter\def\csname PY@tok@bp\endcsname{\def\PY@tc##1{\textcolor[rgb]{0.00,0.50,0.00}{##1}}}
\expandafter\def\csname PY@tok@c1\endcsname{\let\PY@it=\textit\def\PY@tc##1{\textcolor[rgb]{0.25,0.50,0.50}{##1}}}
\expandafter\def\csname PY@tok@kc\endcsname{\let\PY@bf=\textbf\def\PY@tc##1{\textcolor[rgb]{0.00,0.50,0.00}{##1}}}
\expandafter\def\csname PY@tok@c\endcsname{\let\PY@it=\textit\def\PY@tc##1{\textcolor[rgb]{0.25,0.50,0.50}{##1}}}
\expandafter\def\csname PY@tok@mf\endcsname{\def\PY@tc##1{\textcolor[rgb]{0.40,0.40,0.40}{##1}}}
\expandafter\def\csname PY@tok@err\endcsname{\def\PY@bc##1{\setlength{\fboxsep}{0pt}\fcolorbox[rgb]{1.00,0.00,0.00}{1,1,1}{\strut ##1}}}
\expandafter\def\csname PY@tok@kd\endcsname{\let\PY@bf=\textbf\def\PY@tc##1{\textcolor[rgb]{0.00,0.50,0.00}{##1}}}
\expandafter\def\csname PY@tok@ss\endcsname{\def\PY@tc##1{\textcolor[rgb]{0.10,0.09,0.49}{##1}}}
\expandafter\def\csname PY@tok@sr\endcsname{\def\PY@tc##1{\textcolor[rgb]{0.73,0.40,0.53}{##1}}}
\expandafter\def\csname PY@tok@mo\endcsname{\def\PY@tc##1{\textcolor[rgb]{0.40,0.40,0.40}{##1}}}
\expandafter\def\csname PY@tok@kn\endcsname{\let\PY@bf=\textbf\def\PY@tc##1{\textcolor[rgb]{0.00,0.50,0.00}{##1}}}
\expandafter\def\csname PY@tok@mi\endcsname{\def\PY@tc##1{\textcolor[rgb]{0.40,0.40,0.40}{##1}}}
\expandafter\def\csname PY@tok@gp\endcsname{\let\PY@bf=\textbf\def\PY@tc##1{\textcolor[rgb]{0.00,0.00,0.50}{##1}}}
\expandafter\def\csname PY@tok@o\endcsname{\def\PY@tc##1{\textcolor[rgb]{0.40,0.40,0.40}{##1}}}
\expandafter\def\csname PY@tok@kr\endcsname{\let\PY@bf=\textbf\def\PY@tc##1{\textcolor[rgb]{0.00,0.50,0.00}{##1}}}
\expandafter\def\csname PY@tok@s\endcsname{\def\PY@tc##1{\textcolor[rgb]{0.73,0.13,0.13}{##1}}}
\expandafter\def\csname PY@tok@kp\endcsname{\def\PY@tc##1{\textcolor[rgb]{0.00,0.50,0.00}{##1}}}
\expandafter\def\csname PY@tok@w\endcsname{\def\PY@tc##1{\textcolor[rgb]{0.73,0.73,0.73}{##1}}}
\expandafter\def\csname PY@tok@kt\endcsname{\def\PY@tc##1{\textcolor[rgb]{0.69,0.00,0.25}{##1}}}
\expandafter\def\csname PY@tok@ow\endcsname{\let\PY@bf=\textbf\def\PY@tc##1{\textcolor[rgb]{0.67,0.13,1.00}{##1}}}
\expandafter\def\csname PY@tok@sb\endcsname{\def\PY@tc##1{\textcolor[rgb]{0.73,0.13,0.13}{##1}}}
\expandafter\def\csname PY@tok@k\endcsname{\let\PY@bf=\textbf\def\PY@tc##1{\textcolor[rgb]{0.00,0.50,0.00}{##1}}}
\expandafter\def\csname PY@tok@se\endcsname{\let\PY@bf=\textbf\def\PY@tc##1{\textcolor[rgb]{0.73,0.40,0.13}{##1}}}
\expandafter\def\csname PY@tok@sd\endcsname{\let\PY@it=\textit\def\PY@tc##1{\textcolor[rgb]{0.73,0.13,0.13}{##1}}}

\def\PYZbs{\char`\\}
\def\PYZus{\char`\_}
\def\PYZob{\char`\{}
\def\PYZcb{\char`\}}
\def\PYZca{\char`\^}
\def\PYZam{\char`\&}
\def\PYZlt{\char`\<}
\def\PYZgt{\char`\>}
\def\PYZsh{\char`\#}
\def\PYZpc{\char`\%}
\def\PYZdl{\char`\$}
\def\PYZhy{\char`\-}
\def\PYZsq{\char`\'}
\def\PYZdq{\char`\"}
\def\PYZti{\char`\~}
% for compatibility with earlier versions
\def\PYZat{@}
\def\PYZlb{[}
\def\PYZrb{]}
\makeatother


    % Exact colors from NB
    \definecolor{incolor}{rgb}{0.0, 0.0, 0.5}
    \definecolor{outcolor}{rgb}{0.545, 0.0, 0.0}



    
    % Prevent overflowing lines due to hard-to-break entities
    \sloppy 
    % Setup hyperref package
    \hypersetup{
      breaklinks=true,  % so long urls are correctly broken across lines
      colorlinks=true,
      urlcolor=blue,
      linkcolor=darkorange,
      citecolor=darkgreen,
      }
    % Slightly bigger margins than the latex defaults
    
    \geometry{verbose,tmargin=1in,bmargin=1in,lmargin=1in,rmargin=1in}
    
    

    \begin{document}
    
    
    \maketitle
    
    

    
    \section{Introduction}\label{introduction}

Model building and selection has applications across countless fields,
including medicine, biology, and life sciences. The primary interest
often is to identify those most strongly predictive of the outcome via
model building. When datasets are abundant in potential covariates there
may be concern of admitting or retaining possibly meaningless variables.
To account for and control this ``false'' variable selection, Boos et al
(2009) developed the fast false selection rate (FFSR) technique. This is
an algorithm which performs variable selection, model selection, and
model-sizing under the context of forward selection. These three aspects
are determined by controlling for the `false selection rate' (FSR) or
the rate at which unimportant variables are added to the model.

    \subsection{Fast False Selection Rate
Procedure}\label{fast-false-selection-rate-procedure}

Typical forward selection starts with fitting all covariates to
univariate models of the outcome and a single predictor: $Y \sim X_i,$
$i = 1,\ldots,k$. A pre-specified $\alpha$ level is chosen as the cutoff
p-value level for inclusion vs exclusion from the model. The covariate
with the smallest p-value, ``$p$-to-enter'', less than $\alpha$ is kept
and then the process is repeated, with the aforementioned covariate
inclusively fixed in future models. The sequence of p-values for all $k$
covariates is called the \textit{forward addition sequence}.

In the context of forward selection, the FSR algorithm requires a
monotonically increasing forward addition sequence. Hence, if a sequence
is not monotone, the sequence it altered by carrying the largest $p$
forward to give a monotonized sequence of p-values: $\tilde{p}.$ These
are then used to compute the FSR ($\gamma$) level for the model size
corresponding to each $\tilde{p}.$

FSR is defined as
\[ \gamma = \mathrm{E}\left\{\frac{U(Y,X)}{1 + I(Y,X) + U(Y,X)}\right\} \]
where $U(Y,X)$ and $I(Y,X)$ are the number of uninformative and
informative variables in the given model respectively. Hence,
$U + I = S$, the size of the current model. The expected value is with
respect to repeated sampling of the true model, and a 1 is included in
the denominator to avoid division by 0 and account for an intercept. The
goal of the FSR procedure is to pre-specify an initial FSR, $\gamma_0$,
and determine the appropriate $\alpha$-to-enter level to meet this FSR;
i.e.~what must the cutoff ($\alpha$) for any variable to be included in
the model be, in order to restrict the rate at which unimportant
variables enter the model.

Now for a given $\alpha$ the number of uninformative variables in the
model is $U(\alpha) = U(Y,X)$. This quantity is estimated by
$U(\alpha) \approx (k - S(\alpha)) \hat{\theta}(\alpha),$ where
$S(\alpha)$ is the model size at level $\alpha$, $k$ is the total number
of possible predictors, and so $(k - S(\alpha))$ is an estimate of the
number of uninformative variables in the dataset. $\hat{\theta}(\alpha)$
is the estimated rate at which uninformative variables enter the model.
The original FSR method developed by the same authors estimates
$\hat{\theta}$ by simulating `phony' variables (unrelated to the
outcome) and computing the rate at which these enter the model. The new
`fast' method found via simulations that $\hat{\theta}(\alpha) = \alpha$
is an acceptable substitute and considerably faster. It was found to be
produce more accurate predictions than the former FSR when coupled with
a bagging procedure (Boos et al, 2009). Hence, the fast FSR expression
is:
\[ \hat{\gamma}(\alpha) = \frac{(k - S(\alpha))\alpha}{1+S(\alpha)} \]

In this manner one can build a table structured as follows:

\begin{center}
\begin{tabular}{l|l|l|l|l|l}
    \hline
    Size & Variable & p-value & $\tilde{p}$-value & $\hat{\alpha}(\tilde{p})$ & $\hat{\gamma}(\tilde{p})$ \\
    \hline
    1 & V1 & 1e-05 & 1e-05 & 0.002 & 0.0004 \\
    2 & V2 & 0.005 & 0.005 & 0.005 & 0.0120 \\
    3 & V3 & 0.021 & 0.021 & 0.017 & 0.2040 \\
    4 & V4 & 0.009 & 0.021 & 0.028 & 0.2040 \\
    5 & V5 & 0.053 & 0.053 & 0.033 & 0.4241 \\
    \hline
\end{tabular}
\end{center}

where the expression for $\hat{\alpha}$ is
\[ \hat{\alpha}(\tilde{p}) = \frac{\gamma_0(1+S(\tilde{p}))}{k - S(\tilde{p})}\]

In this manner it is easy to select a model size: the size where
$\hat{\gamma} < \gamma_0$. Alternatively, if one utilizes a model of
size $S(\tilde{p})$ with corresponding $\hat{\alpha}(\tilde{p})$ then
one can look to the table to determine what the FSR of the chosen model
is.

The goal of this project was to implement this algorithm under linear
regression in Python. Specifically, functions were be written for the
Fast FSR technique for three contexts: in its simplest form for pure
variable selection, allowing for forced inclusion of a subset of
variables, and with bagging. The code was tested on a dataset available
from NCSU in order to demonstrate the technique and its efficiency. The
algorithm was also compared to an equivalent R version of the algorithm
(Boos, ncsu.edu) on the same dataset to demonstrate the correctness of
the Python function.

    \section{Implementation \& Pseudocode}\label{implementation-pseudocode}

Traditional forward selection can be slow in sizeable datasets. In order
to produce an efficient algorithm, the \texttt{regsubsets} function from
the \texttt{leaps} package in \texttt{R} was utilized to conduct the
forward selection aspect of the FFSR algorithm. This function utilizes a
branch-and-bound technique to quickly identify optimal subsets in the
linear prediction of y by x, rather than fitting all possible models at
each iteration (Lumley, 2015). The remainder of the algorithm was
implemented in a modular programming fashion in Python. These modules
perform the following tasks:

    \subsection{Pseudocode}\label{pseudocode}

\begin{verbatim}
1. Perform forward selection via the `regsubsets` function called in Python via the RPY2 
   library
2. Store covariate order of entry into the model
3. Compute p-values for each sequential covariate entering the model
4. Monotonize p-values by taking sequential max
5. Compute gamma values for each step in model size
6. Compute alpha values for each monotonized p-value
7. Compute alpha for a pre-specified gamma (optional)
8. Estimate parameters for the final fitted model
\end{verbatim}

Additional functions were written to neatly compile the results, as well
as check for appropriate data input type. All of these functions are
called within the primary `ffsr' function. A streamlined version of this
function was written without the data check or table compilation for the
sake of bagging. An additional `bag\_fsr' function was written for
implementation of FFSR with bagging; this function iteratively runs the
ffsr algorithm on a duplicate of the original data built from randomly
selecting the original rows with replacement. In this manner the
resulting parameter estimates can be averaged, akin to model averaging.
This produces predictions more accurate than those obtained with just
one application of the FFSR algorithm (Boos, 2009).

    \section{Testing, Profiling,
Optimization}\label{testing-profiling-optimization}

Unit tests were drafted to assure that functions failed when
appropriate, and raise specific warnings. The functions passed all unit
tests as seens below. The code was also profiled to allow for
optimization to improve efficiency and speed. The primary bottleneck in
the primary ffsr function is within the forward selection modular
function. This procedure requires more than 75\% of the time necessary
to run the functions. Utilizing an \texttt{R} function rather than a
Python or C function appears to be the reason for the slow speed. This
issue is addressed in more detail in the next section.

    \begin{Verbatim}[commandchars=\\\{\}]
{\color{incolor}In [{\color{incolor}1}]:} \PY{k+kn}{import} \PY{n+nn}{os}
        \PY{n}{os}\PY{o}{.}\PY{n}{chdir}\PY{p}{(}\PY{l+s}{\PYZsq{}}\PY{l+s}{/home/bitnami/STA\PYZhy{}663\PYZhy{}Nicole\PYZhy{}Solomon\PYZhy{}Project/Tests}\PY{l+s}{\PYZsq{}}\PY{p}{)}
        \PY{o}{!}py.test
        \PY{n}{os}\PY{o}{.}\PY{n}{chdir}\PY{p}{(}\PY{l+s}{\PYZsq{}}\PY{l+s}{/home/bitnami/STA\PYZhy{}663\PYZhy{}Nicole\PYZhy{}Solomon\PYZhy{}Project/Report}\PY{l+s}{\PYZsq{}}\PY{p}{)}
\end{Verbatim}

    \begin{Verbatim}[commandchars=\\\{\}]
============================= test session starts ==============================
platform linux2 -- Python 2.7.9 -- py-1.4.25 -- pytest-2.6.3
collected 31 items 

test\_alpha.py {\ldots}
test\_alphag.py {\ldots}
test\_bagfsr.py {\ldots}
test\_beta.py {\ldots}
test\_covnames.py .
test\_df\_type.py ..
test\_ffsr.py ..
test\_gamma.py {\ldots}
test\_pvals.py .

{\color{green}}========================== 31 passed in 2.77 seconds ===========================
    \end{Verbatim}

    \section{Application and Comparison}\label{application-and-comparison}

The Python algorithm was applied to a NCAA basketball dataset obtained
from NCSU (Boos, ncsu.edu). The R FFSR algorithm was also applied to
this data in order to compare the efficiency of the Python algorithm.

    \subsection{Standard FFSR function}\label{standard-ffsr-function}

    \begin{Verbatim}[commandchars=\\\{\}]
{\color{incolor}In [{\color{incolor}2}]:} \PY{o}{\PYZpc{}}\PY{k}{run} ffsr\PYZus{}r\PYZus{}run
        \PY{o}{\PYZpc{}}\PY{k}{run} \PYZhy{}t \PYZhy{}m ffsr\PYZus{}p\PYZus{}run
\end{Verbatim}

    \begin{Verbatim}[commandchars=\\\{\}]
NCSU R Results:

    S var   pval  pvmax  alpha      g
1   1   2 0.0000 0.0000 0.0056 0.0000
2   2   3 0.0001 0.0001 0.0088 0.0004
3   5   5 0.0116 0.0116 0.0214 0.0270
4   5   4 0.0053 0.0116 0.0214 0.0270
5   5   7 0.0025 0.0116 0.0214 0.0270
6   6  17 0.0433 0.0433 0.0269 0.0804
7   7  15 0.0527 0.0527 0.0333 0.0791
8  10   6 0.1056 0.1056 0.0611 0.0864
9  10   9 0.0826 0.1056 0.0611 0.0864
10 10   8 0.0536 0.1056 0.0611 0.0864
11 11  12 0.2350 0.2350 0.0750 0.1566
12 12  10 0.2864 0.2864 0.0929 0.1542
13 14  13 0.3163 0.3163 0.1500 0.1054
14 14  18 0.2697 0.3163 0.1500 0.1054
15 15  11 0.4953 0.4953 0.2000 0.1238
16 16   1 0.6326 0.6326 0.2833 0.1116
17 17  14 0.7056 0.7056 0.4500 0.0784
18 18  19 0.8605 0.8605 0.9500 0.0453
19 19  16 0.9032 0.9032    Inf 0.0000
   user  system elapsed 
  0.007   0.000   0.007 

Python Results:

     S  Var       p     p\_m alpha\_F gamma\_F
0    1   x2  0.0000  0.0000  0.0056  0.0000
1    2   x3  0.0001  0.0001  0.0088  0.0004
2    5   x5  0.0116  0.0116  0.0214  0.0270
3    5   x4  0.0053  0.0116  0.0214  0.0270
4    5   x7  0.0025  0.0116  0.0214  0.0270
5    6  x17  0.0433  0.0433  0.0269  0.0804
6    7  x15  0.0527  0.0527  0.0333  0.0791
7   10   x6  0.1056  0.1056  0.0611  0.0864
8   10   x9  0.0826  0.1056  0.0611  0.0864
9   10   x8  0.0536  0.1056  0.0611  0.0864
10  11  x12  0.2350  0.2350  0.0750  0.1566
11  12  x10  0.2864  0.2864  0.0929  0.1542
12  14  x13  0.3163  0.3163  0.1500  0.1054
13  14  x18  0.2697  0.3163  0.1500  0.1054
14  15  x11  0.4953  0.4953  0.2000  0.1238
15  16   x1  0.6326  0.6326  0.2833  0.1116
16  17  x14  0.7056  0.7056  0.4500  0.0784
17  18  x19  0.8605  0.8605  0.9500  0.0453
18  19  x16  0.9032  0.9032  1.0000  0.9032

IPython CPU timings (estimated):
  User   :       0.03 s.
  System :       0.00 s.
Wall time:       0.03 s.
    \end{Verbatim}

    \subsection{FFSR function with forced in
variables}\label{ffsr-function-with-forced-in-variables}

    \begin{Verbatim}[commandchars=\\\{\}]
{\color{incolor}In [{\color{incolor}3}]:} \PY{o}{\PYZpc{}}\PY{k}{run} ffsr\PYZus{}force\PYZus{}r\PYZus{}run
        \PY{o}{\PYZpc{}}\PY{k}{run} \PYZhy{}t \PYZhy{}m ffsr\PYZus{}force\PYZus{}p\PYZus{}run
\end{Verbatim}

    \begin{Verbatim}[commandchars=\\\{\}]
NCSU R Results:

    S var   pval  pvmax  alpha      g
1   3  12 0.0000 0.0000 0.0125 0.0000
2   3   3 0.0000 0.0000 0.0125 0.0000
3   3   5 0.0000 0.0000 0.0125 0.0000
4   4   2 0.0000 0.0000 0.0167 0.0000
5   6   4 0.0043 0.0043 0.0269 0.0079
6   6   7 0.0028 0.0043 0.0269 0.0079
7   8  17 0.0539 0.0539 0.0409 0.0659
8   8  15 0.0458 0.0539 0.0409 0.0659
9  11   6 0.0976 0.0976 0.0750 0.0651
10 11   9 0.0962 0.0976 0.0750 0.0651
11 11   8 0.0281 0.0976 0.0750 0.0651
12 12  10 0.2864 0.2864 0.0929 0.1542
13 14  13 0.3163 0.3163 0.1500 0.1054
14 14  18 0.2697 0.3163 0.1500 0.1054
15 15  11 0.4953 0.4953 0.2000 0.1238
16 16   1 0.6326 0.6326 0.2833 0.1116
17 17  14 0.7056 0.7056 0.4500 0.0784
18 18  19 0.8605 0.8605 0.9500 0.0453
19 19  16 0.9032 0.9032    Inf 0.0000
   user  system elapsed 
  0.009   0.001   0.008 

Python Results:

     S  Var       p     p\_m alpha\_F gamma\_F
0    3  x12  0.0000  0.0000  0.0125  0.0000
1    3   x3  0.0000  0.0000  0.0125  0.0000
2    3   x5  0.0000  0.0000  0.0125  0.0000
3    4   x2  0.0000  0.0000  0.0167  0.0000
4    6   x4  0.0043  0.0043  0.0269  0.0079
5    6   x7  0.0028  0.0043  0.0269  0.0079
6    8  x17  0.0539  0.0539  0.0409  0.0659
7    8  x15  0.0458  0.0539  0.0409  0.0659
8   11   x6  0.0976  0.0976  0.0750  0.0651
9   11   x9  0.0962  0.0976  0.0750  0.0651
10  11   x8  0.0281  0.0976  0.0750  0.0651
11  12  x10  0.2864  0.2864  0.0929  0.1542
12  14  x13  0.3163  0.3163  0.1500  0.1054
13  14  x18  0.2697  0.3163  0.1500  0.1054
14  15  x11  0.4953  0.4953  0.2000  0.1238
15  16   x1  0.6326  0.6326  0.2833  0.1116
16  17  x14  0.7056  0.7056  0.4500  0.0784
17  18  x19  0.8605  0.8605  0.9500  0.0453
18  19  x16  0.9032  0.9032  1.0000  0.9032

IPython CPU timings (estimated):
  User   :       0.04 s.
  System :       0.00 s.
Wall time:       0.04 s.
    \end{Verbatim}

    \subsection{FFSR with bagging}\label{ffsr-with-bagging}

    \begin{Verbatim}[commandchars=\\\{\}]
{\color{incolor}In [{\color{incolor}4}]:} \PY{o}{\PYZpc{}}\PY{k}{run} ffsr\PYZus{}bag\PYZus{}r\PYZus{}run
        \PY{o}{\PYZpc{}}\PY{k}{run} \PYZhy{}t \PYZhy{}m ffsr\PYZus{}bag\PYZus{}p\PYZus{}run
\end{Verbatim}

    \begin{Verbatim}[commandchars=\\\{\}]
NCSU R Results:

   user  system elapsed 
  0.733   0.010   0.749 

[1] "Mean of estimated alpha-to-enter: 0.0454"

[1] "Mean size of selected model: 7.69"

Python Results:


Mean of estimated alpha-to-enter: 0.0503

Mean size of selected model: 7.095

IPython CPU timings (estimated):
  User   :       6.71 s.
  System :       0.27 s.
Wall time:       6.01 s.
    \end{Verbatim}

    As seen in the results above, the Python algorithm yields results
identical to those returned by the R algorithm. The bagging results
differ somewhat due to the random resampling in Python and R that cannot
be matched via setting identical seeds. These results overall
demonstrate the validity of the Python FFSR function. However, the
Python algorithm is significantly slower than the R function. This is
due to the overhead time spent in calling R, running the R function
\texttt{regsubsets}, and then returning the results for
processing.\\This procedure would be significantly faster and
competitive with the equivalent R algorithm if the Fortran code upon
which \texttt{regsubsets} is based were to be wrapped in C and thereby
made directly callable in Python. This is a daunting task however as the
original Fortran code is of the type Fortran 77 and was written decades
ago (Lumley, github.com). At the very least, an alternative, but
performance-wise competitive, forward selection procedure is necessary
to improve the Python algorithm beyond its current speed.

    \section{References}\label{references}

\begin{enumerate}
\def\labelenumi{\arabic{enumi}.}
\itemsep1pt\parskip0pt\parsep0pt
\item
  Boos D, Stefanski L, and Wu Y. ``Fast FSR Variable Selection with
  Applications to Clinical Trials.'' \emph{Biometrics}. 2009; 65(3):
  692-700.
\item
  Boos D and Stefanski L. ``Fast FSR: Controlling the False Selection
  Rate without Simulation.'' NCSU.
  http://www4.stat.ncsu.edu/\textasciitilde{}boos/var.select/fast.fsr.html.
\item
  Thomas Lumley. ``Package `leaps'\,''. R Foundation for Statistical
  Computing, 2015. Version 2.9.
\item
  Thomas Lumley and Alan Miller. ``cran/leaps.''
  https://github.com/cran/leaps/tree/master/src.
\end{enumerate}


    % Add a bibliography block to the postdoc
    
    
    
    \end{document}
